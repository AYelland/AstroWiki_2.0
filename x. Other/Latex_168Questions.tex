% ASTRO Oral Qualification Exam Questions

%%%%%%%%%%%%%%%%% DOCUMENT SETUP %%%%%%%%%%%%%%%%%

% Basic document setup.
\documentclass[11pt, oneside]{book}

% Choose paper style.
% \usepackage[letterpaper, portrait, margin=1in]{geometry}

\usepackage{lineno}
\usepackage{wrapfig}
\usepackage{indentfirst}
\usepackage{mathrsfs}
\usepackage{bbold}
\usepackage{siunitx}
\usepackage{bookmark}
\usepackage{multido}

\usepackage{amsfonts}
\usepackage{amsmath}
\usepackage{amssymb}
\usepackage{amsthm}
\usepackage[english]{babel}
\usepackage{cancel}
\usepackage[font=it, labelfont=it, labelfont=bf]{caption}
\usepackage{comment}
\usepackage{dsfont}
\usepackage{enumitem}
\usepackage{fancyhdr}
\usepackage{float}
\usepackage{framed}
\usepackage[margin=1in]{geometry}
\usepackage{graphicx}
\usepackage{hyperref}
\usepackage[utf8]{inputenc}
\usepackage{mathtools}
\usepackage{multicol}
\usepackage{multirow}
\usepackage{nicematrix}
\usepackage{physics}
\usepackage{scalerel}
\usepackage{setspace}
\usepackage{sidecap}
\usepackage{subfiles}
\usepackage{titlesec}
\usepackage{titletoc}
\usepackage{tensor}
\usepackage{thmtools}
\usepackage{tikz}
\usepackage{tocloft}
\usepackage{xcolor}
\usepackage{xr}

%%%%%%%%%%%%%%%%% USER COMMANDS %%%%%%%%%%%%%%%%%%
% Shortcuts.
\newcommand{\identity}{1\kern-0.25em\text{l}}
\newcommand{\f}[2]{\ensuremath{\frac{#1}{#2}}}
\newcommand{\fpar}[2]{\ensuremath{\left(\frac{#1}{#2}\right)}}
\newcommand{\tfpar}[2]{\ensuremath{\left(\tfrac{#1}{#2}\right)}}
\newcommand{\rd}{\mathrm{d}}
\newcommand{\pd}[2]{\ensuremath{\frac{\partial #1}{\partial #2}}}
\newcommand{\pdd}[2]{\ensuremath{\frac{\partial^{2} #1}{\partial #2^{2}}}}
\newcommand{\td}[2]{\ensuremath{\frac{\mathrm{d} #1}{\mathrm{d} #2}}}
\newcommand{\tdd}[2]{\ensuremath{\frac{\mathrm{d}^{2} #1}{\mathrm{d} #2^{2}}}}
\newcommand{\commute}[2]{\ensuremath{\left[ #1 , \; #2 \right]}}
\newcommand{\acommute}[2]{\ensuremath{\left\{ #1 , \; #2 \right\}}}

\newcommand{\hf}{\ensuremath{\tfrac{1}{2}}}
\newcommand{\hRule}[1]{\rule{\linewidth}{#1}}
\newcommand{\hRightarrow}{\hspace{1cm} \Rightarrow \hspace{1cm}}
\newcommand{\hWhere}{\hspace{1cm} \text{where} \hspace{1cm}}

\newcommand{\relX}{\ensuremath{\vec{\textbf{\textit{x}}}}}
\newcommand{\relP}{\ensuremath{\vec{\textbf{\textit{p}}}}}
\newcommand{\vbar}[1]{\ensuremath{\vec{\vb{#1}}}}

\newcommand{\intInfty}{\ensuremath{\int_{-\infty}^{+\infty}}}
\newcommand{\longdiv}{\smash{\mkern-0.43mu\vstretch{1.31}{\hstretch{.7}{)}}\mkern-5.2mu\vstretch{1.31}{\hstretch{.7}{)}}}}

\newcommand{\fluxunits}{\ensuremath{erg \; s^{-1} \; cm^{-2} \; Hz^{-1}}}

% In-Text highlights.
\newcommand{\definition}[1]{\textcolor{green!50!black}{\textit{``#1''}}}
\newcommand{\important}[1]{\textcolor{black!50!red}{#1}}
\newcommand{\note}[1]{\textcolor{gray}{NOTE: #1}}
\newcommand{\noteContinued}[1]{\textcolor{gray}{#1}}
\newcommand{\solutionColorON}{\color{blue}}
\newcommand{\solutionColorOFF}{\color{black}}
\newcommand{\informationColorON}{\color{black!70!green}}
\newcommand{\informationColorOFF}{\color{black}}
\newcommand{\extraColorON}{\color{gray}}
\newcommand{\extraColorOFF}{\color{black}}

% References command.
\newcommand*{\fullref}[1]{\hyperref[{#1}]{\autoref*{#1}: \nameref*{#1}}}

% Image commands.
\newcommand*\circled[1]{\tikz[baseline=(char.base)]{
            \node[shape=circle,draw,inner sep=2pt] (char) {#1};}}

% Renewing Part/Chapter/Section Commands.
\renewcommand{\thepart}{\Roman{part}}
\renewcommand{\thesection}{\Alph{section}.}

%\pagestyle{headings}

%%%%%%%%%%%%%%%%%%% PAGE SETUP %%%%%%%%%%%%%%%%%%%
% https://tex.stackexchange.com/questions/132170/what-do-headheight-headsep-etc-do-in-the-vmargin-package

% \setlength{\paperwidth}{4.5in} %Paper width.
% \setlength{\paperheight}{8.5in} %Paper height.
% \setlength{\\hoffset}{0in} % 1in + Horizontal Offset.
% \setlength{\voffset}{0in} % 1in + Vertical Offset.
% \addtolength{\oddsidemargin}{-2cm} %Odd Margin.
% \addtolength{\evensidemargin}{-2cm} %Even Margin
% \addtolength{\topmargin}{-0.7in} %Space between Margin & Header (top)
% \addtolength{\headsep}{-0in} %Space between Header (bottom) & Body (top)
\setlength{\headheight}{14pt} %Header Height
% \addtolength{\footskip}{-0.2in} %Space between Footer (bottom) & Body (bottom)
% \addtolength{\textwidth}{1.6in} %Body Width.
% \addtolength{\textheight}{1.6in} %Body Height.

\pagestyle{fancy}
\cfoot{}
\lhead{\textbf{MIT ASTRO Oral Qualification Exam Questions}}
\rhead{\textbf{~Page~{\thepage}}}

\addtolength{\parskip}{\baselineskip} % skips a line between paragraphs
\parindent 0in % no indent at start of paragraph

% \setstretch{1.2}
% \geometry{
%     textheight=9in,
%     textwidth=5.5in,
%     top=1in,
%     headheight=12pt,
%     headsep=25pt,
%     footskip=30pt
% }
\hypersetup{
    colorlinks,
    citecolor=blue,
    filecolor=blue,
    linkcolor=blue,
    urlcolor=blue
}

%%%%%%%%%%%%%%%%%%%%%%%%%%%%%%%%%%%%%%%%%%%%%%%%%%
%%%%%%%%%%%%%%%%%%%%%%%%%%%%%%%%%%%%%%%%%%%%%%%%%%

%%%%%%%%%%%%%%%%% BEGIN DOCUMENT %%%%%%%%%%%%%%%%%

% \bookmarksetup{numbered, open}
% \renewcommand\thepart{\Roman{part}}
% \renewcommand{\thesection}{\arabic{section}}
% \renewcommand{\thesection}{\thesection.\Alph{section}}

\title{
    \textbf{MIT ASTRO Oral Qualification Exam:} \\
    \text{The 168 Questions} \\
    [0.1cm]
    \hRule{1.0pt} \\
    \vspace*{1\baselineskip}
    {\normalsize \color{black} 
    Massachusetts Institute of Technology \\ 
    Astrophysics Division}
    \vspace*{15\baselineskip}}
\author{\textit{(Re-typed and formatted by Alexander Yelland, PhD Candidate)}}
\date{}

\begin{document}
\maketitle 

\pagenumbering{roman}
% \tableofcontents \label{chp:0}

%%%%%%%%%%%%%%%%% BODY OF PAPER %%%%%%%%%%%%%%%%%%
\setcounter{page}{1}
\pagenumbering{arabic}

%%%%%%%%%%%%%%%%%%%%%%%%%%%%%%%%%%%%%%%%%%%%%%%%%%
\section{The Solar System}

\begin{enumerate}[start=1, itemsep=0.4cm]
    \item Describe, qualitatively, the standard model for the formation of the solar system, and discuss observational evidence for this model. Describe observations of exoplanets that have challenged this simple picture as a universal explanation for planet formation.
    \item Explain the steps that you would take to show that both the orbital period and the total energy of a Keplerian orbit depend only on the semi-major axis, and not on the orbital eccentricity.
    \item Calculate the approximate distance to the heliopause. Does the local interstellar medium begin at this boundary? Explain.
    \item Describe the physics involved in the Earth-Moon interaction whereby the Earth's rotation rate is slowing and the orbital separation is increasing.
    \item Describe the Oort cloud and the Kuiper belt, and theories of their origins.
    \item Explain why solar system bodies are often found with integer ratios of orbital periods.
    \item How did the Greeks determine the size of the Earth and the distance to the Moon?
\end{enumerate}

%%%%%%%%%%%%%%%%%%%%%%%%%%%%%%%%%%%%%%%%%%%%%%%%%%
\section{Exoplanets}

\begin{enumerate}[start=8, itemsep=0.4cm]
    \item TESS finds an exoplanet. What can one directly determine from this measurement? What additional observations are needed to determine the mass of this planet?
    \item With what velocity precision must one measure the reflex motion of a sun-like star to detect an Earth-mass planet in a 1-year orbit about it? What astrophysical processes complicate the measurement of radial velocities with this precision?
    \item What is meant by the ``obliquity'' of an exoplanet orbit about its host star? How can obliquity be measured, and what are typical values?
    \item What is meant by tidal locking between an exoplanet and its host star? What are the circumstances where planets are likely to be tidally locked?
    \item Describe qualitatively one of the theories explaining the presence of hot Jupiters -- Jupiter mass planets orbiting at separations of less than 0.05 AU from the star. Why was the discovery of these planets such a surprise?
    \item Describe four methods for detecting an exoplanet, and the potential biases in the inferred populations of planets associated with each method. Roughly how many planets have been detected using each?
    \item Describe a sequence of observations to identify an exoplanet, to determine that it has a rocky composition, and to measure the chemical constituents of its atmosphere. What have we learned about exoplanet interiors and atmospheres from these observations?
    \item Why are M Dwarfs thought to be promising candidates for identifying Earth-sized, temperate (but not necessarily habitable) planets?
\end{enumerate}

%%%%%%%%%%%%%%%%%%%%%%%%%%%%%%%%%%%%%%%%%%%%%%%%%%
\section{Stars and Stellar Evolution}

\begin{enumerate}[start=16, itemsep=0.4cm]
    \item Describe the spectral classification scheme for stars: O, B, A, F, G, K, M. What are the characteristic effective temperatures for stars of each class? What are the characteristic absorption lines observed in the spectra of stars of each class? What are the characteristic luminosities for main-sequence stars of each class? What are the approximate masses for stars in each class?
    \item Sketch the HR diagrams for a typical globular cluster and open cluster. Identify the various observed populations and interpret them on the basis of stellar evolution theory. How would you go about constructing lines of constant stellar radius on the diagram?
    \item From a physics perspective, how does the quantity (B-V) help to determine a star's effective (surface) temperature?
    \item What are stellar populations? Give several examples of Pop. I and Pop. II objects in our galaxy.
    \item What is the ``mass-function'' of a binary star system and how is it determined?
    \item What is meant by the ``gravothermal collapse'' of a globular cluster, and what can save the cluster from complete collapse?
    \item Describe the various evolutionary phases of a low-mass ($1 \; {\rm M_{\odot}}$) star and those of a high-mass (e.g. $12 \; {\rm M_{\odot}}$) star. Show the corresponding evolutionary tracks on an HR diagram.
    \item Describe the types of stellar evolution that lead to type Ia, type Ib and type II supernovae. What are the observational differences among these?
    \item What is a Cepheid variable? Explain the underlying stellar physics involved. What is the Leavitt Law? Over what range of distances do Cepheids play a major role in the ``cosmic distance ladder''.
    \item What are RR Lyrae stars, and how do they differ from Cepheids?
    \item What are HH objects? T Tauri stars? Bipolar flows? OH masers? Where are they all found?
    \item What is a planetary nebula? What is our current understanding of the formation of planetary nebulae?
    \item What is a P Cygni line profile, and what does it signify?
    \item Two stars are observed to have the same color and brightness. One of them is a giant at a greater distance than the other, which is a main sequence star. How could these be distinguished from spectroscopic measurements?
    \item Write down and describe the four basic equations of stellar structure.
    \item Make a dimensional analysis of the equation of hydrostatic equilibrium using a polytropic equation of state to find a general mass-radius relation for spherically-symmetric, self-gravitating bodies. For which two polytropic indices is the configuration unstable?
    \item Make a dimensional analysis of the equation of radiative diffusion in stars to show that the luminosity of a star scales as its mass cubed, if the opacity is taken to be a constant and assuming that radiative diffusion is the dominant mechanism for energy transfer.
    \item Use the known luminosity and mass of the Sun to estimate its nuclear lifetime. What is the current age of the sun, roughly?
    \item Describe the prominent neutrino producing reactions in the sun, and the experiments designed to detect them. What was the solar neutrino problem and how was it solved?
    \item Write down the basic equations of the p-p chain that provides the Sun's nuclear power
    \item How does the CNO cycle work?
    \item Describe the internal structure of the sun. What is the Schwarzschild criterion for convective instability, and how does it delineate the convective and radiative zones in the sun?
\end{enumerate}

%%%%%%%%%%%%%%%%%%%%%%%%%%%%%%%%%%%%%%%%%%%%%%%%%%
\section{Compact Objects and Gravitational Waves}

\begin{enumerate}[start=38, itemsep=0.4cm]
    \item What do we know about the masses and spins of (i) stellar mass black holes and (ii) supermassive black holes, and how do we know these things?
    \item What properties of supermassive black holes correlate with properties of the host galaxy?
    \item It is believed that most stars leave a collapsed remnant at the end of their evolution. What stars leave (i) white dwarfs? (ii) neutron stars? (iii) black holes?
    \item What is a neutron star? What assumptions and inputs go into determining the upper mass limit for a neutron star? What is the approximate ratio of neutrons to protons (and electrons) in the interior of a neutron star?
    \item Why can we not have monopole and dipole terms in gravitational radiation (within Einstein's GR)?
    \item Give a qualitative description of the frequency and amplitude evolution of the gravitational wave signal from a binary compact star system. Compute actual numbers for a $30 \; {\rm M_{\odot}} - 30 \; {\rm M_{\odot}}$ binary black hole system.
    \item A double neutron star system in M31 is about to merge. What is the approximate energy emitted in gravitational radiation and what is the corresponding amplitude (strain) h observed here on Earth? Would LIGO be able to detect it?
    \item Besides the merger of two compact objects, what other sources of gravitational waves are expected (though maybe not yet detected)?
    \item How does the orbital frequency of the innermost stable circular orbit around a black hole scale with its (i) mass? (ii) spin?
    \item What is the Eddington limit and how is it manifested in (i) ordinary stars? (ii) accreting X-ray sources?
    \item What is the Roche potential in a binary system? Describe carefully the assumptions that go into deriving it. Define the Roche limit.
    \item What is the Shakura-Sunyaev (alpha-disk) model for accretion disks? What are the assumptions that go into its derivation?
    \item Write down the fluid equations for conservation of mass and momentum that would describe a spherically symmetric, expanding supernova remnant. How would you derive the ``jump conditions'' for a strong adiabatic shock.
    \item Describe the various stages of evolution of a supernova remnant. What are the relevant physical processes during each phase? Explain why in the Sedov-Taylor phase of a supernova remnant, the radius expands at $t^{2/5}$.
    \item What is a white dwarf star? Why is the radius of a white dwarf a decreasing function of its mass? What is the basic physics that leads to the upper limit on the mass of a white dwarf (i.e. the Chandrasekhar limit)?
    \item What is a ``millisecond pulsar''? What is the shortest spin period known for such an object? Estimate a lower found to its mean density.
    \item Show from a simple dimensional analysis how the effective temperature of an accretion disk depends on accretion rate and distance from the central object.
    \item What evidence is there for ``superluminal'' jets from black holes, and how does one explain the superluminal motion?
    \item What is the historical significance of the Hulse-Taylor binary radio pulsar to physics?
    \item What minimum mass is required for a black hole powering a quasar of luminosity $10^{12} \; {\rm L_{\odot}}$? What accretion rate is required?
    \item How much energy is typically released in a type II supernova? What fractions of that energy are in the forms of neutrinos, visible light, gas kinetic energy and gravitational radiation?
    \item Describe a scenario whereby a neutron star can be formed in a binary system and have the system remain bound.
    \item What is a ``cataclysmic variable''? What are nova explosions, and what is the basic physics underlying these events?
    \item What is the gravitational redshift from the surface of a neutron star?
    \item How much rotational kinetic energy can be stored in a neutron star? How does this help resolve the ``energy budget'' for the Crab nebula?
    \item Explain why you might expect most of the emission of an accreting neutron star to be in the form of X-rays. How does an X-ray pulsar pulse?
    \item Derive a plausible relation between the luminosity of an X-ray pulsar and its spin-up rate. How would you calculate the magnetospheric radius of an accreting neutron star?
    \item Why are more X-ray binaries and radio pulsars found (per unit mass) in globular clusters than in other parts of the galaxy?
    \item What are ``magnetars'', and what evidence do we have that they exist?
    \item Describe the basic features of the ``fireball'' model of gamma-ray burst afterglows. What is the emission mechanism? What inputs are required?
    \item What are the differences between short- and long-duration gamma ray bursts (GRBs)? What do we know about the physical origins of these energetic phenomena?
    \item How would you go about estimating the number of binary systems in the galaxy that contain a black hole? What observational constraints do we have on this number?
    \item How are the magnetic fields of neutron stars estimated in (i) X-ray binaries? (ii) radio pulsars?
    \item Sketch the frequency distribution of the emission of a typical QSO from radio to X-ray frequencies.
\end{enumerate}

%%%%%%%%%%%%%%%%%%%%%%%%%%%%%%%%%%%%%%%%%%%%%%%%%%
\section{ISM/Emission}

\begin{enumerate}[start=72, itemsep=0.4cm]
    \item Discuss the various ``phases'' of gas in the interstellar medium. Are these phases in pressure equilibrium?
    \item What is an HII region? Estimate how the Stromgren radius scales with the luminosity of the ionizing source and with the ambient density.
    \item Explain what bremsstrahlung radiation is. Draw a typical bremsstrahlung spectrum. From what kinds of astrophysical objects is such radiation observed?
    \item What is synchrotron radiation? Draw a typical synchrotron spectrum. From what kind of astrophysical objects is such radiation observed? How can the synchrotron spectrum be used to constrain the age of the source?
    \item What is meant by ``free-free'' absorption? How is this different from electron scattering?
    \item What is the Saha equation and how is it used in stellar structure calculations?
    \item Explain, from a statistical mechanics point of view, why the Balmer lines are most prominent in A stars with an effective temperature of $\approx 10^{4} \; {\rm K}$.
    \item Explain how interstellar dust grains can result in linear polarization of transmitted light. How is the direction of polarization related to the average direction of the interstellar magnetic field (as projected on the plane of the sky)? What major discovery in 2014 was derailed by this phenomenon?
    \item Explain the physics of 21 cm radio emissions from neutral hydrogen atoms.
    \item Qualitatively describe the ``equipartition energy'' of a synchrotron source? What can we learn about cosmic rays from this?
    \item What kinds of sources are detected at TeV energies? What is the ``cosmic gamma ray horizon''?
    \item Why is the gas in the interstellar medium largely transparent at visible wavelengths? How does the transparency of the ISM depend on wavelength in the optical and near-IR, and why?
    \item Sketch a typical cooling function $\Lambda(T)$ for diffuse interstellar gas and identify its prominent features, spanning from IR to X-ray. Overplot a hypothetical heating curve and show how to identify points of thermal equilibrium and their stability.
    \item What is ``brightness temperature''? What is a ``Jansky''? How are these different?
    \item Make a simple, classical argument to show that the spectrum of radiation from monoenergetic electrons with a speed $v$ impinging on ions at an impact parameter $b$ would be roughly flat up to a frequency $\sim v/b$.
    \item If a typical interstellar dust grain is 0.2 microns in size, and starlight suffers an extinction of 1 magnitude per kpc, estimate the space density of dust grains.
    \item What are the Einstein A and B coefficients for a spectral line, and what are the relationships among them?
    \item Explain quantitatively why stimulated emission is important and spontaneous emission is usually ignored in the radio domain, whereas the reverse is true in the optical domain. Given a thermal spectrum at some temperature $T$, at what frequency would the two emission rates be equal?
    \item Name five molecules found in the interstellar medium and comment on how they are detected. What limits our ability to directly detect the most common molecules in the ISM?
\end{enumerate}

%%%%%%%%%%%%%%%%%%%%%%%%%%%%%%%%%%%%%%%%%%%%%%%%%%
\section{Plasma \& Gravitational Lensing}

\begin{enumerate}[start=91, itemsep=0.4cm]
    \item What is ``Faraday rotation''? How is it used in astronomy?
    \item What is ``dispersion measure'' for electromagnetic waves in a plasma? How is it used in astronomy?
    \item What is the physical significance of the plasma frequency?
    \item What are the three types of MHD waves in a magnetized plasma? Are magnetic fields important in the propagation of waves in the interstellar medium? In a star?
    \item What is the ``ideal Ohm's law'' for a plasma? What does ``frozen-in'' mean?
    \item Qualitatively, how does a gravitational lens work? Explain what needs to be measured to use a gravitational lens system to measure the Hubble constant. What is the current status of these determinations of $H_{0}$?
    \item What is the ``Shapiro time delay''? Where was it first measured?
    \item What is gravitational microlensing, what do we learn from it, and how are various experiments studying this phenomenon?
    \item What is the ``weak lensing effect''? How does this differ from ``strong lensing''? What are each of these phenomena useful for to astronomers?
\end{enumerate}

%%%%%%%%%%%%%%%%%%%%%%%%%%%%%%%%%%%%%%%%%%%%%%%%%%
\section{Galaxies}

\begin{enumerate}[start=100, itemsep=0.4cm]
    \item Make a sketch of the Galaxy to scale (top and side views) and indicate its various features and properties. How are stars, gas, dust, and dark matter distributed in our Galaxy?
    \item What observational evidence do we have for dark matter in galaxies and clusters of galaxies?
    \item What are the typical mass ranges of dwarf galaxies, normal galaxies, and galaxy clusters? Describe a few ways in which we measure the masses of these different types of systems.
    \item Sketch the rotation curve of our Galaxy, with approximate scales on the axes. How can information about the rotation curve be determined from 21 cm observations?
    \item What is the density profile of a self-gravitating isothermal gas sphere? What is the corresponding rotation curve for this system?
    \item What is two-body relaxation? For a self-gravitating cluster of $N$ objects, each of mass $m$, with a velocity dispersion $\sigma$, what is the relaxation time? How long does it take for a massive object ($M \gg m$) to sink to the bottom of a cluster potential well?
    \item What is the ``Local Group''?
    \item What is the morphology-density relation for galaxies? Give several possible explanations for this trend.
    \item What is the ``Schechter luminosity function''? What is the luminosity of a typical bright galaxy? How does the luminosity function of galaxies compare to the mass function of halos from simulations?
    \item Describe Hubble's classification scheme for galaxies and explain why it is useful.
    \item What is the ``Tully-Fisher relation''? Derive this relation beginning with the virial theorem and using simplifying assumptions about galaxies. How can this relation be used to determine $H_{0}$?
    \item What are cD galaxies and where are they found? How might they be formed?
    \item What is the ``Faber-Jackson relation'' for elliptical galaxies? How is this different than the ``fundamental plane'' for elliptical galaxies?
    \item What is a violent relaxation? How does the phase space distribution function it produces differ from that of an isothermal gas? How does the timescale for violent relation compare to that for weak 2-body interactions?
    \item Why do some galaxies have prominent spiral structure and others do not? Briefly describe the density wave theory of spiral structure?
    \item How many globular clusters does our Galaxy contain? How are they distributed in space? What fraction of the total mass of the Galaxy do they contain?
    \item A globular cluster at a distance of 10 kpc, containing 106 stars, subtends an angle of 1/3 arcminute. Estimate the velocity dispersion among its stars.
    \item What are the Oort A and B coefficients and what basic information about the Galaxy can be determined from them? What are the four types of observations that go into constraining these coefficients, and how precisely are each measured by current technology.
    \item Define the following simple ``laws'' and ``profiles'' and for what galaxy component or galaxy type each is most relevant: Sérsic, de Vaucouleurs, exponential, Einasto, NFW. Describe where observational data deviates from these idealized profiles.
    \item What is ``Press-Schechter theory'' and why is it wrong?
    \item What is ``biased galaxy formation''?
    \item What fractions of each of the following is composed of ``dark matter'': (i) The solar neighborhood, (ii) a typical galaxy like the Milky Way, (iii) a typical galaxy cluster like the Coma cluster, (iv) The universe. How are each of these estimates arrived at?
    \item What is the galaxy correlation function? How do the correlation functions of galaxies and clusters of galaxies differ? How is the galaxy correlation function used to constrain $H_{0}$?
    \item What are ``active galactic nuclei''? What is the ``unified model of AGN'', and how does it explain the variety of AGN that are observed? How and when are AGN important for galaxy evolution?
    \item Describe the following relations and what they are useful for: (i) Kennicutt-Schmidt; (ii) M-sigma relation; (iii) the Madau plot; (iv) the star-forming main sequence.
    \item Describe the following galaxy-formation problems: (i) the missing satellite problem; (ii) the core- cusp problem; (iii) the cooling flow problem. Which of these are solved? What are their solutions?
\end{enumerate}

%%%%%%%%%%%%%%%%%%%%%%%%%%%%%%%%%%%%%%%%%%%%%%%%%%
\section{Cosmology}

\begin{enumerate}[start=126, itemsep=0.4cm]
    \item Summarize the state of the experimental field of 21cm cosmology and what it hopes to accomplish.
    \item Summarize the state of the experimental CMB field and what it hopes to accomplish.
    \item What are the anisotropies in the CMB? How do their amplitudes depend on angular scale?
    \item What are acoustic peaks in the CMB power spectrum? What do their locations and amplitudes tell us?
    \item Write down the Jeans equation for a disturbance propagating in a self-gravitating medium. From this show how to find the critical wavenumber for propagating modes. What is the Jeans mass?
    \item Qualitatively, how does a density fluctuation grow over cosmic history and how does the answer differ for dark matter and baryons?
    \item What is the ``spherical top hat'' model for the formation of galaxies and clusters? What is the significance of the number $10 \pi^{2}$?
    \item Summarize the observational evidence in favor of the standard ``Big Bang'' model of the universe.
    \item What is the Hubble constant? Explain in detail at least three independent methods to measure it. How does it relate to the age of the universe?
    \item How can the age of the universe be estimated empirically? Do the current estimates agree with that obtained from the Hubble constant?
    \item How would you compute the age of our universe using the Friedmann equation? What approximations can be made to simply this calculation?
    \item What is the Robertson-Walker metric? What are the Friedmann equations? How does the universe expand if it is (i) radiation dominated? (ii) matter dominated with $\Omega \ll 1$, $\Omega = 1$, $\Omega \gg 1$? (iii) dominated by a cosmological constant?
    \item What was the approximate temperature of the universe at recombination, and why did recombination happen then?
    \item When (or at what redshift) did the universe become (i) matter dominated? (ii) optically thin to electron scattering? What temperature was the CMB at these times? What significance did these events have for the CMB?
    \item What evidence is there for dark energy and what might it be?
    \item Is most of the hydrogen in the universe neutral or ionized? Describe the approximate ionization history of the Universe, and at what redshift the major phase transitions occurred.
    \item What is the Ly alpha forest? Why are cosmologists interested in it?
    \item What are the ``flatness problem'' and the ``isotropy problem'' in standard Big-Bang cosmology? How does the inflationary model resolve these problems?
    \item What is the baryonic contribution to the cosmological mass density and how is it determined? Where are the bulk of the baryons in our Universe?
    \item How did the particle content of the universe evolve? What is the most abundant particle in the universe today?
    \item Describe, qualitatively, the synthesis of light elements in the Big Bang. What is the deuterium bottleneck, and how does it help produce the right helium abundance? What role does radiation play in Big Bang nucleosynthesis?
    \item What are the thermal and kinetic Sunyaev-Zel'dovich effects? What are they useful for?
    \item What range of physical scales are being probed by current and future CMB experiments? Are the anisotropies related to galaxy formation? What is the current observational status?
    \item What is the ``Lyman limit'', and how does it relate to observations of high-redshift galaxies?
    \item What mass should a neutrino have to close the universe? Compare it with current upper bounds (or detections).
    \item What is the meaning and purpose of a $\log N - \log S$ curve? Explain the current interpretation of this curve for gamma-ray bursts.
    \item In our Universe, at roughly what scale does perturbation theory break down and become nonperturbative?
\end{enumerate}

%%%%%%%%%%%%%%%%%%%%%%%%%%%%%%%%%%%%%%%%%%%%%%%%%%
\section{Instrumentation \& Astronomical Methods}

\begin{enumerate}[start=153, itemsep=0.4cm]
    \item Explain how a radio interferometer manages to produce sub-arcsecond images when none of the constituent dishes has an angular resolution of better than 1 arcmin.
    \item How does an X-ray telescope image? Why is it necessary to have the reflection take place at grazing angles of incidence?
    \item How would you calculate the ``signal-to-noise ratio'' for a standard ground-based telescope performing CCD imaging? What terms dominate at optical versus infrared wavelengths? Why, when observing objects that are fainter than physical foregrounds like the night sky, does SNR increase as $sqrt{\rm time}$?
    \item What is the diffraction limit for a telescope? Why does the signal-to-noise ratio scale like $D^{4}$ for diffraction limited imaging on ground-based telescopes with diameter $D$ at IR wavelengths?
    \item Briefly describe the following kinds of astronomical instruments and their purpose: Schmidt camera, Ritchy-Chretien telescope, multi-object spectrograph, echelle spectrograph, CCD, coronograph, laser interferometer, adaptive optics.
    \item What limits the angular resolution of ground-based telescopes at $1 \; {\rm \mu m}$, $10 \; {\rm \mu m}$, or $100 \; {\rm m}$?
    \item What are ``apparent magnitude'', ``absolute magnitude'', and ``bolometric magnitude''? What are U, B and V colors?
    \item Briefly describe the main objectives and capabilities of the following astronomical observatories Voyager, HST, Magellan telescopes, Chandra, WMAP, MWA, SWIFT, Planck, ROSAT, Spitzer, TESS, LIGO, LISA, NICER, CHIME, HERA, WINTER, SPT, JWST, Roman Space Telescope, Rubin Observatory.
    \item What are the pros/cons of having a space-based versus ground-based observatory? At what wavelengths can you do both? At what wavelengths can you only observe from space?
    \item Derive the approximate (within a factor of a few) spectral resolution $R$ required to detect a planet that perturbs its host star by a radial velocity amplitude of 10 cm/s. Describe some practical challenges with calibration and stability, and how they are addressed for such instruments. Why have we not yet achieved this precision in RV measurements of real stars?
    \item Summarize the current state of experimental gravitational wave detection, and the prospects for improvement in the near future.
\end{enumerate}

%%%%%%%%%%%%%%%%%%%%%%%%%%%%%%%%%%%%%%%%%%%%%%%%%%
\section{Lightning Round / Miscellany}

\begin{enumerate}[start=164, itemsep=0.4cm]
    \item What kind of objects would you guess the following were? HD 128220, Abell 426, QSO 1013+277, a Crucis, Mk 509, 3C 234, NGC 2808, PSR 0950+08, BD +61° 1211, Beta Pictoris b.
    \item What are the strongest spectral lines seen in each of the following: Integrated light from a typical galaxy, a quasar, a 100 km/s interstellar shockwave, a giant molecular cloud?
    \item How far away are the following objects: The Sun, the Orion Nebula, M13, M31, the Virgo Cluster, the Coma Cluster, 3C 273? How large and how massive are they?
    \item What are the brightest extra-solar-system sources at the following wavelengths: radio, mm, mid- IR, optical, UV, X-ray, gamma ray? Which, if any, are isotropic on the sky?
    \item Briefly describe the following famous astronomical objects: The Crab, M3, M31, M87, W51, Cyg X-1, 3C 273, Cyg A, SS 433, GW150914, GW170817 , LMC, Boötes Void, Virgo cluster, TRAPPIST-1, Prox Cen b, Sgr A*.
\end{enumerate}

%%%%%%%%%%%%%%%%%%%%%%%%%%%%%%%%%%%%%%%%%%%%%%%%%%
\end{document}
